\documentclass[singlecolumn,english,aps,prl,showpacs,floatfix,superscriptaddress,notitlepage]{revtex4-1}
\usepackage[T1]{fontenc}
\usepackage[latin9]{inputenc}
\setcounter{secnumdepth}{3}
\usepackage{amsmath}
\usepackage{microtype}
\usepackage{dcolumn}
\usepackage{graphicx}
\usepackage[small,bf,hang,raggedright,FIGTOPCAP]{subfigure}
\usepackage{varioref}
\usepackage{epstopdf}
\usepackage{color}
\usepackage{eucal}
\usepackage{float}
\usepackage{soul}
\graphicspath{{./pictures/}}
\usepackage{braket}
\usepackage{blkarray}

\DeclareMathOperator{\Tr}{Tr}

\makeatletter

\usepackage{babel}

\begin{document}

\title{Notes: interpolation of the electron-phonon coupling}

\maketitle

This is a summary of the strategy for interpolation adopted by Phoebe, as detailed also in Ref. [Giustino PRB  76, 165108 (2007)].

The coupling to be used for the calculation of scattering rates is:

\begin{equation}
g^{SE}_{mn,\nu} (k,q) = \bigg( \frac{1}{2 m \omega_{q\nu}} \bigg)^{1/2} g_{mn,\nu} (k,q)
\end{equation}

What we need to interpolate instead is:
\begin{equation}
g_{mn,\nu} (k,q) = \bra{mk+q} \partial_{q\nu}V \ket{nk}
\end{equation}

First recall the relation between the wavefunction in Wannier and Bloch representation.

\begin{equation}
\ket{mR_e} = \sum_{nk} e^{-ikR_e} U_{nm,k} \ket{nk}
\end{equation}

\begin{equation}
\ket{nk} = \frac{1}{N_e} \sum_{mR_e} e^{ikR_e} U_{mn,k}^\dagger \ket{mR_e}
\end{equation}
where $N_e$ is the number of supercells.



Let $e_{q\kappa}^{\nu}$ be the phonon eigenvector we get from diagonalizing the dynamical matrix.
We define $u_{q\kappa}^{\nu} = (\frac{m_0}{m_{\kappa}})^{1/2} e_{q\kappa}^{\nu}$, where $m_0$ is the electron mass, and $m_{\kappa}$ is the ionic mass.

To transform the potential from the reciprocal to the real space representation, we have:
\begin{equation}
\partial_{\kappa R_p} V(r)
=
\frac{1}{N_p}
\sum_{q\nu} e^{-iqR_p} [u_{q\kappa}^{\nu}]^{-1} \partial_{q\nu} V(r)
\end{equation}


So, we first transform to Wannier space by:
\begin{equation}
g(R_e,R_p)
=
\frac{1}{N_p}
\sum_{kq} e^{-ikR_e-iqR_p} U_{k+q}^\dagger g(k,q) U_k u_q^{-1}
\end{equation}

Then, we interpolate to Bloch space

\begin{equation}
g(k,q)
=
\frac{1}{N_e}
\sum_{R_e R_p} e^{ikR_e+iqR_p} U_{k+q} g(R_e,R_p) U_k^\dagger u_q
\end{equation}


Details:
the mesh of k and q points must be the same (or at least commensurate, so that we can map k+q into the same k grid).

grids must be unshifted, so that k+q falls into the same grid.
Also, if k+q is outside the 1st BZ, we can use the parallelt transport gauge and set $U_{k+q+G} = U_{k+q}$.

q is in the irreducible wedge, k is on the full grid.


\end{document}
